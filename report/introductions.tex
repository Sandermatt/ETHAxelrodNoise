\section{Introduction and Motivation}

\label{sec:intro}

\subsection{The Prisoner's Dilemma}
The prisoner's dilemma is a model from game theory. $2$ people are suspected to have done a crime together. Now they are examined separately in different rooms. In this situation, they can either whistle-blowing the other person to protect oneself or keep silent. Over all, it is of advantage, if both keep silent. But for the single person it is better to betray the other person. The risk of betraying is the following: if both accused people betray the other, the penalty for both is the highest. This problem is in game theory called "Prisoner's dilemma" \cite{stanford}.

\subsection{The Axelrod Experiment}
In the year $1981$, Robert Axelrod invited for a competition to the iterated prisoner's dilemma. Iterated in this context means there are played an arbitrary number of games against the same opponent. Each player therefor knows all his own decisions and the decisions of his opponent. So the goal is not just to betray once, but to keep the own reward as high as possible. \\

People from different fields like mathematics, politics, economy or psychology have been asked to develop a winning strategy for this competition. All the different strategies were playing against another to find the most successive strategy. Interestingly, the very simple strategy "Tit for Tat" (\verb0TFT0) won the tournament. During the first round, \verb0TFT0 keeps silent (cooperation) and during the rest of the game, just does, what its counter player did the round before.\\

This sort of experiment is very interesting, because the results can be applied in many different fields in real life. Just one out of many examples: $2$ countries make an agreement on their amount of weapons. For the single country it is of advantage to haves more military strength than the other nation. But as in the prisoner's dilemma, if both nations rise their military strength, for both it is just a loss money and an increase in danger. \cite{axelrod}

\subsection{Introduction of Noise}
A further development in the Axelrod Experiment is the introduction of noise. This means, cooperation is wrongly understood as defection and vise versa. The introduction of noise to the axelrod experiment is nothing new, but very important, because in real world, noise and small distortions are always present. This can lead to serious complications. An example therefor: \textit{"On September 1,1983 a South Korean airliner mistakenly
flew over the Soviet Union (Hersh 1989). It was shot down by the Soviets, killing all 269
people aboard. The Americans and Soviets echoed their anger at each other in a short,
but sharp escalation of cold war tensions."}\cite{wu}\\

There are a lot more of example like the one above. So there are some questions concerning noise in an iterated prisoner's dilemma like situation. Can a dispute based on miscommunication be overcome? Is there a way, treason can be hidden behind pretended miscommunication? Does the miscommunication even discourages cooperation? How much miscommunication can cooperation survive? Do learning strategies have an advantage over the other ones and how do the traditional players act? And last but not least, how does the final result change, if noise is introduced?