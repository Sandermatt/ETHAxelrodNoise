\section{Summary and Outlook}
The aim of this simulation was to investigate two things. In the first place was the impact of noise on the tournament. It turned out that noise, which lets defections appear as cooperations is beneficial for most players. The opposite, when cooperative moves are perceived as defections, has a much larger impact, although negative for most of the players. The performance of most friendly players drastically drops. It is especially harsh for players that relied on the effect of the first move being cooperative and have no mechanism to restore cooperation once it is lost. Even the more defective players lose some of their reward in comparison with no noise, but it is not as dramatically as for the cooperative players. So the noise definitely discourages cooperation of all players. Further cooperation is vulnerable to noise and can not really overcome the miscommunication. Generally one could say, whether a transmitted cooperation is for real or is just misunderstood does not really matter. It is better for everyone to see a corrupted cooperation than none at all, so treason can be hidden behind pretended miscommunication pretty well.\\

The second investigated topic was, how learning players would perform in this setup. Of the three learning mechanisms is just copying the others. \verb0EVO0 and \verb0SS0 are based on this strategy, whereat \verb0SS0 performed strongest. The strategy was even stronger than most not learning strategies. The other two approaches failed, because they were not responsive. Learning strategies have no big advantage over the none evolutionary strategies, because they rely on data corrupted by noise.\\

To overcome a dispute based on miscommunication is very difficult and is risky, because of the exploitation by the opponent. To overcome a dispute both players have to take some risk and be reconcile to get back to mutual cooperation again. The best case is achieved if both players are reconcile, so the chance to get back to mutual cooperation after some noisy decision is the biggest.\\

Continuative there are some interesting things more to do. The performance of the players was heavily impacted by the nature of the other players participating in the simulation. It would be interesting to run the simulation with more players or even every player against one specific player, to see how good he can adapt. Another possible investigation could be to find out if the average performance increases with a noise that covers up defections forever, or if there is a turning point after which the performance decreases again. In addition the concept of evolutionary strategies can be augmented.
