\section{Submitted Researchplan}
\subsection{General Introduction}
Tournament like simulation of the prisoner's dilemma with repeated inter-
actions. Random errors are introduced in the information about the player's 
recent behavior. We want to observe the different outcome of the traditional
players if noise is introduced. Further we want to try to implement new 
players with learning strategies. \\
We believe that this makes the simulation more realistic.\\
Extension of Axelrod's Tournaments.

\subsection{Fundamental Questions}
Can a dispute based on miscommunication be overcome?\\
Can treason be hidden behind pretended miscommunication?\\
Does miscommunication discourage cooperation?\\
How much miscommunication can cooperation survive?\\
Do learning strategies have an advantage over the other ones?\\
How do the traditional players act and how does the final result change, 
if noise is introduced?\\
Independent variables: length of simulation, reliability of communication, rewards\\
Dependent variables: correlation between cooperation and success, frequency of cooperation, successful strategies

\subsection{Expected Results}
Miscommunication works against cooperating strategies.\\ 
Programs that reconcile are more successful.\\
The reward of the learning players is less influenced by the noise.

\subsection{References}

\begin{itemize}
\item On Evolving Robust Strategies for Iterated Prisoner's Dilemma, P. J. DARWEN and X. YAO, 16. November 1993\\
\item Multiagent Reinforcement Learning in the Iterated Prisoner's Dilemma, T. W. SANDHOLM and R. H. CRITES\\
\item Adaptation of Iterated Prisoner's Dilemma Strategies by Evolution and Learning, H. Y. QUEK and C. K. GOH, 2007\\
\item How to Cope with Noise in the Iterated Prisoner's Dilemma, J. WU and R. AXELROD, JOURNAL OF CONFLIC RTESOLUTION, Vol. 39 No. 1, March1995 183-189\\
\item Five Rules for the Evolution of Cooperation, M. A. NOWAK, Science 314, 1560 (2006)\\
\end{itemize}
\subsubsection{Research Methods}

Agent-Based Model

\subsection{Other}
The type(s) of the learning strategies we will decide later, after reading some of the literature.

\clearpage

\section{Matlabcode}

\subsection{Master.m}

\matlabscript{../matlab/Master}{Master.m}

\subsection{win.m}

\matlabscript{../matlab/win}{win.m}

\subsection{show\underline\ data.m}

\matlabscript{../matlab/show_data}{show\underline\ data.m}

\subsection{playerlist.m}

\matlabscript{../matlab/playerlist}{playerlist.m}

\subsection{player1.m}

\matlabscript{../matlab/player1}{player1.m}

\subsection{player2.m}

\matlabscript{../matlab/player2}{player2.m}

\subsection{player3.m}

\matlabscript{../matlab/player3}{player3.m}

\subsection{player4.m}

\matlabscript{../matlab/player4}{player4.m}

\subsection{player5.m}

\matlabscript{../matlab/player5}{player5.m}

\subsection{player6.m}

\matlabscript{../matlab/player6}{player6.m}

\subsection{player7.m}

\matlabscript{../matlab/player7}{player7.m}

\subsection{player8.m}

\matlabscript{../matlab/player8}{player8.m}

\subsection{player9.m}

\matlabscript{../matlab/player9}{player9.m}

\subsection{player10.m}

\matlabscript{../matlab/player10}{player10.m}

\subsection{player11.m}

\matlabscript{../matlab/player11}{player11.m}

\subsection{player12.m}

\matlabscript{../matlab/player12}{player12.m}

\subsection{player13.m}

\matlabscript{../matlab/player13}{player13.m}

\subsection{player14.m}

\matlabscript{../matlab/player14}{player14.m}

\subsection{player15.m}

\matlabscript{../matlab/player15}{player15.m}

\subsection{player16.m}

\matlabscript{../matlab/player16}{player16.m}

\subsection{player17.m}

\matlabscript{../matlab/player17}{player17.m}

\subsection{player18.m}

\matlabscript{../matlab/player18}{player18.m}

\subsection{player19.m}

\matlabscript{../matlab/player19}{player19.m}

\subsection{player20.m}

\matlabscript{../matlab/player20}{player20.m}




